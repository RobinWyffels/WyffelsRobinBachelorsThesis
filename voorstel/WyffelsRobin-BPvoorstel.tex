%==============================================================================
% Sjabloon onderzoeksvoorstel bachproef
%==============================================================================
% Gebaseerd op document class `hogent-article'
% zie <https://github.com/HoGentTIN/latex-hogent-article>

% Voor een voorstel in het Engels: voeg de documentclass-optie [english] toe.
% Let op: kan enkel na toestemming van de bachelorproefcoördinator!
% \documentclass[english]{hogent-article}
\documentclass{hogent-article}

% Invoegen bibliografiebestand
\addbibresource{voorstel.bib}

% Informatie over de opleiding, het vak en soort opdracht
\studyprogramme{Professionele bachelor toegepaste informatica}
\course{Bachelor's thesis}
\assignmenttype{Research proposal}

\academicyear{2025-2026}

\title{Evaluating Sim-to-Real Performance of an Autonomous Warehouse Robot Trained in Virtual Environments Using Isaac Sim and ROS 2}

\author{Wyffels Robin}
\email{wyffels.robin@student.hogent.be}

\supervisor[Co-promotor]{W. Martens (\href{https://www.mrwatts.io/}{Mr. Watts NV}, \href{mailto:wouter.martens@mrwatts.io}{wouter.martens@mrwatts.io})}

\specialisation{Robotics \& Embedded Systems Development}
\keywords{Robotics, Sim-to-Real, Autonomous Warehouse Robot, Isaac Sim, ROS 2}

\begin{document}

\begin{abstract}
Deze bachelorproef onderzoekt de sim-to-real prestaties van een autonome magazijnrobot die volledig virtueel is getraind met NVIDIA Isaac Sim en ROS 2. Het centrale vraagstuk is hoe goed het gedrag en de prestaties van een robot, ontwikkeld in een digitale omgeving, overeenkomen met zijn functioneren in een echte magazijncontext.

De robot wordt vanaf nul ontworpen en gebouwd, inclusief hardware (chassis, aandrijving, hefmechanisme) en softwaremodules (ROS 2-nodes voor communicatie en besturing). In de simulatieomgeving worden navigatie, mapping en obstakelvermijding getraind en getest, waarna dezelfde taken fysiek worden uitgevoerd in een gecontroleerde magazijnsetting. De prestaties worden kwantitatief vergeleken op metrics zoals taakduur, afgelegde afstand, botsingen en kaartnauwkeurigheid.

Verwacht wordt dat er een duidelijke kloof ontstaat tussen simulatie en realiteit, veroorzaakt door sensorruis, variabele lichtomstandigheden en moeilijk te simuleren factoren zoals vloerfrictie. Toch biedt de simulatie-first aanpak aanzienlijke voordelen: kortere ontwikkeltijd, lagere kosten en snellere iteraties. De meerwaarde voor IT-professionals en bedrijven ligt in praktische richtlijnen voor virtuele training, inzicht in beperkingen van sim-to-real overdracht en aanbevelingen voor het optimaliseren van magazijnautomatisering.
\end{abstract}

\tableofcontents

%---------- Inleiding ---------------------------------------------------------

\section{Introduction}%
\label{sec:introduction}

the automatization of warehouses in recent years has started playing an increasingly important role in industries where efficiency, cost saving and safety are central. Autonomous robots were traditionally developed and tested in real life environments, which is time-consuming and expensive. An important question therefore is: How well does a robot trained in a virtual environment fare in the real world. This will be the central focus of this bachelor's thesis.

The case focusses on designing, building and programming an autonomous robot that can mo- ve boxes in a warehouse. The robot will be developed completely from scratch, including: hardware (chassis, propulsion, lifting mechanism) and software (ROS 2 nodes for internal systems communication, external communication and control systems). The innovative concept is the use of NVIDIA Isaac Sim and Isaac Lab to train the models and virtually simulate the robot. This bachelors thesis targets IT-Professionals and industrial automation engineers, as well as specific companies that are interested in optimizing warehouse automation technologies. 

The problem statement is as follows: 
How well does an autonomous robot, trained in a virtual environment fare in a real world warehouse environment? 
The central research question is: 
How well does the behaviour and performance of a virtually trained autonomous robot translate in a real world warehouse environment?

The goal of this research is to realise a proof of concept: 
A working robot that can autonomously … tasks in a warehouse where the behaviour and performance will be measured against the simulated environment. 
The end result includes a functional prototype, an analysis of the differences between simulated and real world performance, and the findings of using NVIDIA Isaac Sim and Isaac Lab for virtual training of autonomous industrial robots. 


%---------- Stand van zaken ---------------------------------------------------

\section{Literature review}%
\label{sec:Literature review}
% TODO add sourses

\subsection{Intro: Warehouse Automation and Autonomous Mobile Robots (AMRs)}
The automation of warehouses is rapidly growing due to the use of AI-driven systems like AMRs, robotic manipulators, and AGV's (Automated Guided Vehicle). 
Recent specialized literature highlights that AI techniques, including machine learning, computer vision, and reinforcement learning, enhance the accuracy and cost efficiency of intralogistics. 
However, implementation complexity and real-world validation remain critical challenges, particularly within dynamic warehouse environments characterized by changing layouts and human interaction. 
The current findings indicate a simulation-first approach where robot behaviors and policies are first developed and tested in a virtual environment before making their way into physical pilots. 
AMR-specific warehouse operation reviews reference questions about navigation without fixed infrastructure, safe human-robot interactions, and scalable software-stacks, which further proves the need for research into sim-to-real performance % TODO add citations.

\subsection{Virtual training and simulation first robot development}
Virtual training in hyper realistic environments are generally seen as a way of increasing safety, accelerating iterations, while decreasing cost. 
Peer-reviewed research shows that VR/simulation can train robots in complex scenarios, provided that the transfer to the real world is carefully evaluated. 
The Navigation2 stack (Nav2) In the ROS ecosystem, illustrates that modern navigation architectures (behaviour trees, dynamic planners/controllers) can be deployed in unscripted, busy environments (Marathon2 experiment), but also that tuning and robustness remain important issues. 
This precisely demonstrates the need for semantically linking simulation experiments to real world measurements.

\subsection{NVIDIA Isaac ecosystem: Isaac Sim, Isaac Lab, Isaac Gym}
Isaac sim (Omniverse) is a physics-based, GPU-accelerated simulation with RTX-sensorrendering, synthetic data (replicator), Ros 2-bridges and workflows for SIL/HIL. This offers the possibility of making a digital twin of warehouse scenes and validating robot stacks for real world tests. 
Isaac lab offers an open modular framework for robot learning (RL, lfD, motion planning) with domain randomization and large scale efficiency simulations. Relevant in methodically reducing the sim-to-real gap. Historically Isaac Gym marked a breakthrough: a completely GPU-based pipeline that accelerates RL-training by an order of 2-3 times by keeping physics and policy training on the same GPU; this makes large scale parallel training of tasks like locomotion or manipulation possible and ideal for exploring real world deployments.

\subsection{Ros 2 as the backbone of modern autonomous robots}
ROS 2 forms an industrial middleware with DDS-based communication, lifecycle management, and better safety pillars than ROS 1. The Science Robotics publication describes its design, architecture, and applications “in the wild,” which legitimizes ROS 2 for safety-critical settings such as warehouses. On top of ROS 2, Navigation2 offers a modern navigation stack (behavior trees, planners/controllers) which, in the Marathon 2 experiment, functioned reliably and without failure in crowded campus scenarios over a long period. For an applied-IT case, ROS 2 is therefore a robust foundation for integrating sensors, actuation, planning, and AI policies—and for gradually replacing classic modules with learned components.

\subsection{Warehouse specific insights and research gaps}
The AMR review for warehouses concludes that time/effort waste is mainly in material transport, which AMRs can reduce by collaborating with humans and taking over repetitive tasks; at the same time, it highlights needs around safety, control software, and navigation in dynamic layouts without infrastructure. Broad reviews of AI warehouse automation emphasize the trend toward edge computing, RL, and computer vision, but warn that validation in the real world remains crucial to support simulation-based claims. This places our case (training virtually, then comparing physically) right at the heart of the current gap: quantitatively measuring how well virtual policies perform on real floors with variable lighting conditions, friction, and sensor noise.

% Voor literatuurverwijzingen zijn er twee belangrijke commando's:
% \autocite{KEY} => (Auteur, jaartal) Gebruik dit als de naam van de auteur
%   geen onderdeel is van de zin.
% \textcite{KEY} => Auteur (jaartal)  Gebruik dit als de auteursnaam wel een
%   functie heeft in de zin (bv. ``Uit onderzoek door Doll & Hill (1954) bleek
%   ...'')


%---------- Methodologie ------------------------------------------------------
\section{Methodology}%
\label{sec:Methodology}

something here

%---------- Verwachte resultaten ----------------------------------------------
\section{Expected result, conclusion}%
\label{sec:Expected_result_conclusion}


\nocite{Macenski2022,OpenNavigation2025,NVIDIA2025,IsaacLab2025,Makoviychuk2021}
\printbibliography[heading=bibintoc,section=0]

\end{document}